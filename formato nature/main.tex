%%%%%%%%%%%%%%%%%%%%%%% file template.tex %%%%%%%%%%%%%%%%%%%%%%%%%
%
% This is a template file for Web of Conferences Journal
%
% Copy it to a new file with a new name and use it as the basis
% for your article
%
%%%%%%%%%%%%%%%%%%%%%%%%%% EDP Science %%%%%%%%%%%%%%%%%%%%%%%%%%%%
%

\documentclass[twocolumn, A4]{webofc}
%%% "twocolumn" for typesetting an article in two columns format (default one column)
%%%\documentclass{webofc}

\usepackage[varg]{txfonts}   % Web of Conferences font
%
% Put here some packages required or/and some personnal commands
%
\usepackage{amsmath,amssymb}
\usepackage{caption}
\usepackage{subcaption}
\usepackage{lineno}
%
\begin{document}
%
\title{Diseño de birrefringentes mediante placas de onda compuestas}
%
% subtitle is optionnal
%
%%%\subtitle{Do you have a subtitle?\\ If so, write it here}

\author{\firstname{Cristian Eduardo} \lastname{Hernández Cely}\inst{1}\fnsep\thanks{\email{cristian.hernandez17@correo.uis.edu.co}} \and \firstname{Jhon} \lastname{Pabon}\inst{1}  \and \firstname{Brayan} \lastname{Pedraza}\inst{1} \and
        \firstname{Rafael Ángel} \lastname{Torres Amaris}\inst{1}\fnsep\thanks{\email{Rafael.Torres@saber.uis.edu.co}}
           % etc.
}

\institute{Universidad Industrial de Santander}

\abstract{Resumen}
%
\maketitle
\linenumbers
\section{Introduction}\label{intro} 

%Waveplates
Waveplates como materiales birrefringentes para controlar la polarización,

la no idempotencia es explotada en el diseño para mediente Composite Waveplates diseñar retardadores con modos propios y birrefringencias ajustables

%CW, estado del arte

%Teorema I de Jones

%Fibras ópticas como una CW
Puede usarse estas expresiones para describir medios complejos tales como fibras ópticas modeladas como CW. 

Este artículo se divide de la siguiente forma: 

\section{Depolarization in birefringent media}\label{sec-1}
    
   
\section{Results}\label{EaR}
   Composite waveplate 

\section{Discusion}\label{Dis}
Mediante la no idempotencia en el operador birrefringencia, cuando ambas rotan sincronizadamente ya que en general no se puede describir la superposición de varias láminas birrefringentes con modos lineales  ahora su operador es un birrefringente elíptico. Esta no idempotencia puede ser considerada en la descripción de otras configuraciones de composite waveplates o en sistemas de múltiplas capas birrefringentes. De donde se concluye que en un CW, el número de placas así como su orden deben considerarse para su caracterización o explorarse como grados de libertad para generar otras variantes en las configuraciones.

\section{Conclusions}
 
%
\bibliographystyle{woc}	
\bibliography{References.bib}

\end{document}

% end of file template.tex